\section{Introduction and Overview}


\subsection{Motivation}
I want a job, or something. 

\subsection{Scope of the thesis}
Clarify what you want to cover. 

Batteries are vastly complex and much efforts have been devoted to the development of these. Yet, with all these efforts, it still is a never ending chase for batteries that can push the limits of their properties even further. This work proposes a methodology to predict these properties accurately without the need of big scale simulations, or computer heavy calculations. Using state of the art machine learning, and base properties of all ready existing databases, we prose a set of predictors to see if we can predict the properties of new, undiscovered electrodes, or even new properties in already well known electrodes.

\subsubsection{Research Question}

How to better batteries?

RQ1: Is there potential for the use machine learning to easy the search for better battery materials?

RQ2: Which ML method would be the most optimal for such a search?

RQ3: What predictors are the most suited for such a task, and which would yield the most efficient training.

RQ4: Does the size of the database matter? Or is it possible to find a solution with a "not optimal", or even small, database?

\subsubsection{Approach}

The choice of features examined in this work is inspired by an extensive survey done on similar project especially in the field of metal organic framework done by my supervisors in Crete, and dictated, to some degree by the lack of more data.

In order to evaluate the effect of different features, a prediction approach using principle component analysis was utilized. First we decided to use physical descriptors, that is for instance geometrical properties of the unit cell. This because it was greatly efficient in other studies(REF), and is straightforward. 

Then we needed to find other descriptors, and the void fraction seemed like the next obvious thing

\subsection{Structure of the thesis}

First the most essential concepts from the fields of batteries, machine learning, and work already explored on these two fields conjoined, are introduced. Then the method will be explained before rounding up our results so far before trying to put this all into perspective.