\section{Result section title}
%MSE, RMSE, MAE , PCA, R2, compair.
%Ordered by target:

The accuracy of the ML method in this work is evaluated using R-squared ($R^2$), "Mean" of the training target values (y), standard deviation (stdev) of the error, mean absolute error (MAE), root-mean-square error (RMSE) and weighted absolute percentages error (WAPE), as described in \ref{sec:evaluation_method}. Lower RMSE, MAE, and WAPE, with a high $R^2$ indicates a better ML predictions. 
Before each run a PCA is done and the singular values that collectively accounted for $99\%$ of the variation are selected for the RF run, the last $1\%$ of the variation, (i.e. the other predictors) are removed.

For each intercalation battery type; both Mg- and Li-, four different sets of descriptors are reported, namely;

\begin{itemize}
\item Material specific properties
\item Volumetric number density 
\item Void fraction
\item APRDF
\end{itemize}


The targets the suggested features are tested on are; Average Voltage (\ac{AV}), Gravimetric Capacity (\ac{GC}), Volumetric Capacity (\ac{VC}), Specific Energy (\ac{SE}), and Energy Density (\ac{ED}). 
After all sets of descriptors were tested, they were combined to make the best possible classifier.

The two data sets are chosen much due to their size. The Mg-ion database (\ac{db}) has $356$ different batteries, while the Li-ion db has $2081$ different Li batteries, more than five times the size of the Mg-ion db. Both of these databases are relative small form a ML perspective, but are big enough to show correlations. 

There are two hyper parameters that need to be adjusted when using random forest. These are the number of estimators and the maximum number of features. The number of estimators are the number of decision trees used in the forest. The optimal amount of trees used for the model is a compromise between predicational* quality and computational cost: The more trees, the better the prediction becomes, but the cost is that the model becomes more computationally expensive. However, the improvement rate will slow down after a critical number of trees. To find the optimal number of trees for our predictions, we made several model using  $5,10,25,50,100,250,500,1000,2500,5000,10000$ decision trees, creating $10$ unique forests for each amount of trees and plottet them with the standard deviation of the $10$ runs. The results, for both databases can be seen in figure \ref{fig:trees_size}. We decided to use $250$ estimators, this gave a good compromise between cost and accuracy.  For the latter, our random forest regressor always considers all features instead of a random subset of features. 

Impact of the size of our dataset was also tested using $10\%,20\%,...,100\%$ of our database. Our data were split into two parts, the training data $X_{\text{train}}, 80\%$, and the test data, $X_{\text{test}},20\%$, for testing of our model. These calculations were also done $10$ times for each unique percentage. The mean of $R^2$CVM and WAPE was plottet with their respective standard deviations against the percentage of the database used, as seen in figure \ref{fig:trees_size}. 

\begin{figure}[H]
    \centering
    \begin{subfigure}{0.45\textwidth}
        \centering
        \includegraphics[width=\linewidth]{result/Mg_n_estimators.png}
        \caption{Mg}
        \label{fig:n_estimators}
    \end{subfigure}%
    ~ 
        \begin{subfigure}{0.45\textwidth}
        \centering
        \includegraphics[width=\linewidth]{result/Mg_size_db.png}
        \caption{Mg}
        \label{fig:size_db}
    \end{subfigure}
    ~ 
        \begin{subfigure}{0.45\textwidth}
        \centering
        \includegraphics[width=\linewidth]{result/Li_number_of_estimators.png}
        \caption{Li}
        \label{fig:size_db}
    \end{subfigure}
    ~ 
        \begin{subfigure}{0.45\textwidth}
        \centering
        \includegraphics[width=\linewidth]{result/Li_size_db.png}
        \caption{Li}
        \label{fig:size_db}
    \end{subfigure}
	\caption{a) and c) Shows the number of estimators used plottet vs. the predictions and WAPE. b) and d) Indicates the impact of the size of the database, with percentage of the database plottet vs. the accuracy of the predictions and WAPE. The predictions are done with the same features as in table \ref{tab:Li-vnd-msp-vf-stability}, with Average Voltage as the target.}
	\label{fig:trees_size}
\end{figure}

 
%run with smaller vs bigger db of same ion. 
 
% Write a paragraph on how the PCA made it easy to remove input data that were not relevant for the ML, due to the information already existing. 
 
 
  
\subsection{Material specific properties}
The "material specific properties" (\ac{msp}), the properties related to every individual material, both charged and discharged, where tested first. They are; energy, energy per atom, volume of the unit cell, band gap, density, magnetization, number of sites, and elasticity, for both the charged and discharged material, as explained in \ref{sec:theory_bat}. The Materials Project database included "formation energy per atom" for most structures, but due to "None" values and no significant change in our predictions, probably due to no new variance, we decided to leave this predictor out of the result section. The results for msp can be seen in table \ref{tab:mg-msp} and in table \ref{tab:Li-msp}.

\begin{table}[H]
\scriptsize
\title{Mg-db on msp}
\centering
\caption{msp prediction-scores for the Mg db tested against the given targets.}
\begin{tabular}{|c|c|c|c|c|c|}
	\hline 
	$\frac{Target: \rightarrow}{Accuracy:\downarrow} $ & AV & GC & VC & SE & ED 
	 \\ 
	\hline
	$R^2$-score & 0.5224 & 0.4023 & 0.47064 & 0.4997 &  0.4826\\ 
	\hline 
	$R^2$-train & 0.9293 & 0.9208 & 0.9078 & 0.9317 &  0.90721 \\ 
	\hline
	Mean: &2.5830	&194.105633	&844.3028&503.0492&2105.3873	\\
	\hline 
	Stdev:&0.2905	&22.8769	&103.3817	&79.9453	&399.8795	\\
	\hline
	RMSE:&0.2907& 22.9191& 103.3826 & 80.0723 & 400.4032 \\ 
	\hline 
	MAE: &0.2042& 16.7839 &  74.45455 & 61.0444 & 279.353 \\ 
	\hline
	WAPE: & 7.9079 & 8.6468 & 8.8184  & 12.1348	& 13.2685 \\
	\hline
	$R^2$ CVM: &  0.60989 & 0.471065 & 0.4985  & 0.56822 &0.5247 \\
	\hline
	components used: & 10/16 & 10/16 & 10/16  & 9/16 &9/16 \\
	\hline
\end{tabular}
\label{tab:mg-msp}
\end{table}

From Table \ref{tab:mg-msp}, it is clear that msp are correlated to the targets, with especially high values for AV. The accuracy falls and uncertainty for the predictions rises in capacity related targets, and for SE and ED. The WAPE also indicates that the uncertainty is high for SE and ED. It is obvious that we either lack part/s of the picture, due to $R^2$-train score being so low (not closer to $100\%$, which shows that the ML algorithm can not make a perfect regressor even when it knows the answers). Or that our data contains to much noise. It is likely to be the case for all our runs, because we do not know what predictors are needed. The msp seems to be a considerable part of the puzzle and will be used in later combinations of predictors.

\begin{table}[H]
\scriptsize
\title{Li-db on msp}
\centering
\caption{msp from the Li-db tested against the given targets.}
\begin{tabular}{|c|c|c|c|c|c|}
	\hline 
	$\frac{Target: \rightarrow}{Accuracy:\downarrow} $ & AV & GC & VC & SE & ED  \\ 
	\hline
	$R^2$-score 	& 0.5384 & 0.4162 & 0.4870 & 0.3888 &  0.44016\\ 
	\hline 
	$R^2$-train 	&  0.9308 & 0.9146 & 0.9277 & 0.9189 & 0.9185 \\ 
	\hline
	Mean: 	 &3.58038&132.6524&463.5728&479.7717&1609.3506\\
	\hline 
	Stdev:	 &0.2853	&21.7816	&70.5023	&75.2707&272.7973	\\
	\hline
	RMSE: 	& 0.2853 & 21.804& 70.5346 & 75.323 & 735.50848 \\ 
	\hline 
	MAE: 	&0.1969& 15.3558 &  48.2923 & 55.3367 & 191.3195 \\ 
	\hline
	WAPE: 	& 5.5013 & 11.5759 & 10.4174  & 11.5339 & 11.8879 \\
	\hline
	$R^2$ CVM: &  0.5692 & 0.4456 & 0.5136  & 0.4572 &0.4815 \\
	\hline
	components used: & 9/16 & 9/16 & 9/16  & 8/16 &9/16 \\
	\hline
\end{tabular}
\label{tab:Li-msp}
\end{table}

When applying the same predictors on the Li- battery db we observe a fall in prediction accuracy on all targets except VC, but the uncertainty of our predictions rises for both GC and VC. For AV SE and ED our weighted absolute percentage error sinkes considerably, and thus the results are more likely to indicate the correlation between our predictor and the target more accurately.  
The increase in data seems to not effect our predictions to a large extent. The $R^2$-train is increased by a negligible amount. 

\subsection{Volumetric number density}
\subsubsection*{Mg-ion framework}

Volumetric number density (vnd) as described (\ref{sec:volumetric_number_density}) are shown, first for the Mg-database than for the Li-database. Due to the nature of vnd, and our database having both charged and discharged materials, it is necessary to try the three alternatives; only the charged materials, only the discharged materials  and the combination of both the charged and the discharged materials.

\begin{table}[h]
\scriptsize
\centering
\caption{Mg- db, the charged material as vnd predictors. A total of 21 components are applicable. Mg-prediction results on the targets; Average Voltage (AV), gravimetric capacity (GV), volumetric capacity (VC), specific energy (SE), and energy density (ED). Each row showes the number representing that type of test, as included in section (\ref{sec:evaluation_method}). } 
\title{Mg-db on n, charged materials}
\begin{tabular}{|c|c|c|c|c|c|}
	\hline 
	$\frac{Target: \rightarrow}{Accuracy:\downarrow} $ & AV & GC & VC & SE & ED 
	 \\ 
	\hline
	$R^2$-score 	& 0.51920 & 0.1783 	& 0.2513 & 0.5751 &   0.2864\\ 
	\hline 
	$R^2$-train 	& 0.94365 & 0.9055 	& 0.91827 & 0.9194 & 0.9302 \\ 
	\hline
	Mean: 		& 2.5492 &192.7605	&848.22	&537.5&2227.422\\
	\hline 
	Stdev: 		& 0.2604 &24.9649	&98.056&84.4191&335.7547\\
	\hline		
	RMSE:		 & 0.26043& 24.965 &  98.0653 &  84.5840 &336.2162 \\ 
	\hline
	MAE: 		& 0.1860 & 51.9769& 70.8905 & 63.5310 & 252.03396 \\ 
	\hline
	WAPE:		& 7.2986 & 9.2791 & 8.3575  & 11.8200 & 11.3150 \\
	\hline
	$R^2$ CVM: 	& 0.58126 &  0.2464 & 0.3964  & 0.5280 &0.5174 \\
	\hline
	components :	& 19/21 & 19/21 & 19/21  & 19/21 &19/21 \\
	\hline
\end{tabular}
\label{tab:mg-n-i}
\end{table}


\begin{table}[h]
\scriptsize
\centering
\caption{Mg-db, the discharged material as vnd-predictors. A total of 30 components are applicable. Predictions on the targets; Average Voltage (AV), gravimetric capacity (GV), volumetric capacity (VC), specific energy (SE), and energy density (ED). Each row showes the number representing that type of test, as included in section (\ref{sec:evaluation_method}).}
\title{Mg database on n, the discharged materials}
\begin{tabular}{|c|c|c|c|c|c|}
	\hline 
	$\frac{Target: \rightarrow}{Accuracy:\downarrow} $ & AV & GC & VC & SE & ED 
	 \\ 
	\hline
	$R^2$-score 	& 0.4879 & 0.7213 & 0.5941 & 0.4612 &  0.4779\\ 
	\hline 
	$R^2$-train 	& 0.9312 &  0.92499 & 0.9332 & 0.9496 & 0.9218 \\ 
	\hline
	Mean: 	 	&2.7038	&184.0281&869.2605&483.4859& 2288.429	\\
	\hline 
	Stdev:		&0.2835	&24.0464	& 86.9195	&75.7915& 352.595	\\
	\hline		
	RMSE: 		&0.2836& 24.04749 &  86.9622 & 75.8086 & 352.6381 \\ 
	\hline
	MAE: 		& 0.1938 & 14.4620& 44.5180 & 57.4124 & 221.0466 \\ 
	\hline
	WAPE: 		& 7.1708 & 7.8586 & 5.1213  & 11.8746 & 9.6593 \\
	\hline
	$R^2$ CVM: 	& 0.5885 & 0.6190 & 0.64017  & 0.6115 &  0.5980 \\
	\hline
	components: 	& 22/30 & 22/30 & 22/30  & 22/30 &23/30 \\
	\hline
\end{tabular}
\label{tab:mg-n-ii}
\end{table}


\begin{table}[h]
\scriptsize
\centering
\caption{Mg-db using both the discharged- and the charged materials as vnd-predictors. A total of 51 components are applicable. Predictions on the targets; Average Voltage (AV), gravimetric capacity (GV), volumetric capacity (VC), specific energy (SE), and energy density (ED). Each row showes the number representing that type of test, as included in section (\ref{sec:evaluation_method}).}
\title{Mg database on vnd, both charged and discharged}
\begin{tabular}{|c|c|c|c|c|c|}
	\hline 
	$\frac{Target: \rightarrow}{Accuracy:\downarrow} $ & AV & GC & VC & SE & ED 
	 \\ 
	\hline
	$R^2$-score 	& 0.6094 & 0.6150 & 0.7993 & 0.5685 &  0.6560\\ 
	\hline 
	$R^2$-train 	& 0.9507 & 0.9402 & 0.9452 & 0.9540 &  0.9385 \\ 
	\hline
	Mean: 	 	& 2.7871	&194.4788&825.5211&483.6478& 2166.943\\
	\hline 
	Stdev:	 	& 0.2405	&20.3516	&80.1226 	&69.5891	& 314.8012\\
	\hline 
	RMSE: 		&0.2409& 20.3608 &  80.1511 & 69.59 &314.8037\\ 
	\hline
	MAE: 		& 0.1857 & 11.9768& 38.6913 & 52.7039 & 219.9826 \\ 
	\hline
	WAPE: 		& 6.6656 & 6.1584 & 4.6868  & 10.8971 & 10.1517 \\
	\hline
	$R^2$ CVM: 	& 0.6261 	& 0.6622 	& 0.6758  	& 0.64942 &0.6532 \\
	\hline
	components: 	& 30/51 	& 27/51 	& 29/51 	 & 29/51 	&27/51 \\
	\hline
\end{tabular}
\label{tab:mg-n-iii}
\end{table}


There are a couple of different results that are particularly interesting. First of all; The use of all vnd-predictors does not yield the best predictions in all cases. There are targets that respond better to the use of one state of material, charged or discharged, than the other. Gravimetric and volumetric capacity responds particularly well to the discharged materials only, but improve ( by a small percentage  $3-5\%$) when given the combination of materials. AV are approximately the same (change of $0.7\%$), while SE and ED improve by $9\%$. This seems reasonable due to the discharged materials containing more information than the charged materials (i.e. the battery framework $\ce{Mg_{0-1}CrF_6}$ with $\ce{CrF_6}$ being the charged material and  $\ce{MgCrF_6}$ being the discharged material).

As expected, the combination of both the charged and discharged materials give the best results, and as can be seen from the number of components used, a lot of information overlapp between the discharged and charged materials. 

When combining the two group of predictors our predictions on all five targets are between $62$ to $68$ percent, with high certainty. Which showes that this is a good predictor, and is definitely a part of the physical picture. Noticeably the $R^2$-train is also considerably better (than for msp) when including the specific atoms as we do in vnd.  

As expected the PCA tells us that there is an overlap in the information from the charged and discharged materials as given by the number of components used by the ML algorithm. i.e. for charged materials on AV 17/21 components where used to account for $99\%$ of the variance in the data, for the discharged materials 19/30 components where used, while when combined only 26 out of the total 51 components were necessary to achieve the same $99\%$. This indicates a information overlap between the predictors. 

Lastly, the quality of our estimator drops significantly for specific energy and energy density. It seems that the quality of our predictor drops on these predictors. (SP $- \si{Wh/kg}$, ED $- \si{Wh/l}$). \myworries{why?}


\subsubsection*{Predictions on Li-ion intercalation frameworks with vnd}
The same technique, as used in the previous section, was applied to the mp Li-ion db, with $2108$ battery frameworks instead of $365$, as were the case for Mg-ion intercalation type db. This is a bigger database with different and more unique atoms, which is likely to calls for more variability and possibly more bias, and most likely a lower uncertainty in the predictions. 

First of all, it is no obvious improvement when using only the discharged materials, as were with the Mg db. The charged materials seems to be a better predictor for energy density, and average voltage. While the discharged materials are better at predicting capacity. When predicting the specific energy they seems to be equal in their predictions. 
If we combining the two, we quickly see that the variation in the data behaves in the same way as for the Mg db, there is an obvious overlap in information and correlation in the predictors. Volumetric capacity, with the overall best predictions ($71.86\%$), only uses $38$ components, out of the possible $107$ (to account for $99\%$ of the variance). When using the charged and discharged predictors the machine learning algorithm used $38/54$ components for the charged material and $34/53$ components for the discharged components. It stands to reason that volumetric capacity would be a good target for vnd due to the intrinsic relation to volume.


Average voltage predictions are $56\%$ with high certainty, lower than the predictions for the Mg-ion db, but a bit higher accuracy. The gravimetric capacity and specific energy are lower in the Li-ion db, with higher accuracy. While energy density is approximately equal and equally accurate. 


Onward, both the charged and discharged group of predictors, for both db, will be used in further predictions. 


\begin{table}[H]
\scriptsize
\title{Li-db on vnd - charged}
\centering
\caption{Li - vnd charged.}
\begin{tabular}{|c|c|c|c|c|c|}
	\hline 
	$\frac{Target: \rightarrow}{Accuracy:\downarrow} $ & AV & GC & VC & SE & ED 
	 \\ 
	\hline
	$R^2$-score 	& 0.4466 & 0.22431 & 0.4248 	&  0.40232 &  0.3402\\ 
	\hline 
	$R^2$-train 	& 0.9321 & 0.9060 	& 0.9170 	& 0.9166 	& 0.9287 \\ 
	\hline
	Mean: 		&3.589	&134.03	&467.40	&480.46	&1628.3	\\
	\hline 
	Stdev:		&0.2884	&22.165	&73.304	&75.481	&241.6	\\
	\hline
	RMSE: 		& 0.2888 & 22.174	& 73.313 	& 75.538	 & 241.8 \\ 
	\hline 
	MAE: 		& 0.1952 & 15.371	 & 49.472	 & 53.076	 & 176.38 \\ 
	\hline
	WAPE: 		& 5.440 	&  11.469 	& 10.584  	& 11.047 	&10.833 \\
	\hline
	$R^2$ CVM: 	& 0.5437 & 0.3692 	& 0.4352  	& 0.4324 	&0.4690 \\
	\hline
	components used: & 37/54 & 41/54 & 38/54  & 38/54 &35/54 \\
	\hline
\end{tabular}
\label{tab:Li-vnd-i}
\end{table}

\begin{table}[H]
\scriptsize
\title{Li-db on vnd - discharged}
\centering
\caption{Li- vnd discharged.}
\begin{tabular}{|c|c|c|c|c|c|}
	\hline 
	$\frac{Target: \rightarrow}{Accuracy:\downarrow} $ & AV & GC & VC & SE & ED 
	 \\ 
	\hline
	$R^2$-score 	& 0.5099 &  0.3731 & 0.4539 & 0.3133 &  0.3519\\ 
	\hline 
	$R^2$-train  	 & 0.9269 & 0.9163 & 0.9248 & 0.9178 & 0.9122 \\ 
	\hline
	Mean:	 	 &3.5455	& 133.12&476.6359&473.7458	&1609.5469\\
	\hline 
	Stdev:		 &0.2858	&21.237&70.9524&76.4492&292.0871\\
	\hline
	RMSE:		 & 0.2860 & 21.2419& 70.95331 & 76.4636 & 292.1071 \\ 
	\hline 
	MAE:	 	&0.1895& 14.1890 &  45.2840 & 54.9314 & 177.8518 \\ 
	\hline
	WAPE: 		& 5.3454 & 10.659 & 9.501  & 11.5951 &11.0498 \\
	\hline
	$R^2$ CVM: &  0.5319 & 0.3983 & 0.4724  &  0.4367 &0.44020 \\
	\hline
	components used: & 40/53 &  38/53 &  34/53  &  39/53 & 39/53 \\
	\hline
\end{tabular}
\label{tab:Li-vnd-i}
\end{table}

\begin{table}[H]
\scriptsize
\title{Li-db on vnd - charged and discharged}
\centering
\caption{Li- vnd both charged and discharged}
\begin{tabular}{|c|c|c|c|c|c|}
	\hline 
	$\frac{Target: \rightarrow}{Accuracy:\downarrow} $ & AV & GC & VC & SE & ED 
	 \\ 
	\hline
	$R^2$-score 	& 0.4755 & 0.54190 & 0.6923 & 0.5388 &  0.5912\\ 
	\hline 
	$R^2$-train 	& 0.9355 & 0.94517 & 0.9550 &   0.9351 & 0.9476 \\ 
	\hline
	Mean: 		&3.616	&135.22&460.05&473.75	&1643.15	\\
	\hline 
	Stdev:	 	&0.2657	&17.383	&54.816	&69.721	&213.12	\\
	\hline
	RMSE: 		& 0.2661 & 17.390	& 54.830 & 69.779	 & 213.17 \\ 
	\hline 
	MAE: 		&0.1859& 11.089	 &  32.248 & 46.392 & 144.95 \\ 
	\hline
	WAPE: 		& 5.141 & 8.201 	&  7.0095  & 9.793 &8.822 \\
	\hline
	$R^2$ CVM: 	&  0.5608 & 0.6191 & 0.7186  & 0.5619 &0.6506 \\
	\hline
	components used: & 39/107 & 37/107 & 38/107  & 38/107 &40/107 \\
	\hline
\end{tabular}
\label{tab:Li-vnd-iii}
\end{table}






%$n$ is overall better at predicting VC and with that GC naturally follows. In the first figure(\ref{tab:mg-n}), the Mg-database, every prediction has a lower 'score' than in the Li-database. This is probably due to it being a much smaller database. 

%\myworries{Should do better calculations on this, maybe when results and Battery-section is done.}

%As can be seen clearly; the evaluations of this method is somewhat splitt. The MSE is generally better for the Li-database, and is best at the AV. The CV is better for the capacities but there is somewhat a high degree of uncertainty. This is probably due to the database being small. As expected the results for volumetric capacity is the peak of these runs. 

%One can from these results conclude that volumetric number density is worth bringing on as a predictor, as it clearly is a part of the puzle.  





\subsection{Void fraction}
Results for predictions using only the void fraction methods predictors, that is, the charged and discharged materials, helium volume and the charged and discharged geometric (point accessible) volume. First on the Mg-ion db, then on the Li-ion db.

%\subsubsection*{Predictions on Mg-ion intercalation frameworks with the void fraction predictors}

\begin{table}[H]
\scriptsize
\centering
\caption{Mg- db prediction on the targets AV, GC, VC, SE, ED. A total of 4 predictors where used in this run.}
\title{Mg database on voidfraction}
\begin{tabular}{|c|c|c|c|c|c|}
	\hline 
	$\frac{Target: \rightarrow}{Accuracy:\downarrow} $ & AV & GC & VC & SE & ED 
	 \\ 
	\hline
	$R^2$-score & -0.3240 & 0.2858 & 0.2387 & 0.12553 &  0.06999\\ 
	\hline 
	$R^2$-train & 0.8621 &  0.9047 & 0.9266 & 0.8821 & 0.89021 \\ 
	\hline
	Mean: 	& 2.657	&185.0915&830.5563& 492.9507	&2181.7746\\
	\hline 
	Stdev:	&0.4195	&27.3274	&93.0447&111.29816	&420.6466\\
	\hline 
	RMSE:	 &0.4197& 27.3296 &  93.07669 & 111.3178 & 420.6466 \\ 
	\hline
	MAE: 	& 0.3229 & 19.6535& 68.1477 & 88.04975 & 324.1471 \\ 
	\hline
	WAPE: 	& 12.1522 & 10.6182 & 8.20506  & 17.8617 & 14.85704 \\
	\hline
	$R^2$ CVM: & 0.06214 &  0.3490 & 0.3943  & 0.09092 & 0.18253 \\
	\hline
	components: & 3/4 	& 4/4 	& 4/4 	 & 3/4	 & 3/4 \\
	\hline
\end{tabular}
\label{tab:mg-n-ii}
\end{table}

As a predictor for the magnesium db, void fraction seems to be inaccurate for most targets, with the best accuracy score of $39.4\%$ on volumetric capacity. Second best prediction is on gravimetric capacity ($34.9\%$), which is very close related. The other predictions have to low correlation to consider them as predictors, so AV, SE and ED will be omitted for the rest of this subsection.


%\subsubsection*{Predictions on Li-ion intercalation frameworks with the void fraction predictors}

On the Li-ion db the same trend as on Mg-ion db seems persistent. More data seems to emphasize that the void fraction is not a good descriptor for the given targets. The best predictions are still on VC, followed by GC, but the scores are not high enough to indicate a proper correlation. 

\begin{table}[H]
\scriptsize
\centering
\caption{Li- db prediction on the targets AV, GC, VC, SE, ED. A total of 4 predictors where used in this run.}
\title{Li database on void fraction}
\begin{tabular}{|c|c|c|c|c|c|}
	\hline 
	$\frac{Target: \rightarrow}{Accuracy:\downarrow} $ & AV & GC & VC & SE & ED 
	 \\ 
	\hline
	$R^2$-score 	& -0.0373 & 0.0603 & 0.05857 &  -0.1374 &  0.01655\\ 
	\hline 
	$R^2$-train 	&  0.8510 &  0.87095 & 0.89077 &  0.8601 & 0.8669 \\ 
	\hline
	Mean: 		 &3.5262&133.9714&447.216	&457.3083&1636.7998	\\
	\hline 
	Stdev:		 &0.3914	&26.2527	&90.6257&102.7326&345.827\\
	\hline 
	RMSE: 		&0.0455& 26.253 &  90.6261 & 102.74241 & 345.8724 \\ 
	\hline
	MAE: 		& 0.3028 & 19.6381& 66.8044 & 76.8657 & 261.30785 \\ 
	\hline
	WAPE: 		& 8.5886 & 14.6584 & 14.9378  & 16.8082 & 15.9645 \\
	\hline
	$R^2$ CVM: & -0.06318 & 0.05186 & 0.159419  & -0.02663 & 0.02518 \\
	\hline
	components: & 3/4 	& 4/4 & 4/4  & 4/4 & 4/4 \\
	\hline
\end{tabular}
\label{tab:Li-poreBlazer}
\end{table}


On a general note, in all runs, three to four feature were needed to account for $99\%$ of the variability. Between $80-90\%$ of the variability can be traced back to one predictor, and yet it is only possible to get a $R^2$-CVM prediction above $0\%$ for any target, by including all 4 predictors. This points at void fraction, as we have approached it here, being a bad descriptor for the given targets. 

It seems reasonable that $90\%$ of the variability are in one predictor due to these 4 predictors measuring the same physical thing, this physical feature seems to be expressed by having some correlation with the volumetric capacity, but this correlation might be covered by noise as we introduce more data.

Void fraction was tested as a predictor for VC, and other targets, in combination with other predictors, but due to drops in predictive capability the results are omitted from the result section. These results can be found on github.

%----------------------------------------------------------------------------
\subsection{AP-RDF cross-product approach}

The AP-RDF was first tested with a cross-product approach, this is explained in the method section \ref{meth:AP-RDF1}. As is clear form the table (\ref{tab:Mg-APRDF1}) the first approach to AP-RDF did not give a good correlation with the targets and the error is to big. The idea was that it is better for a RF approach to consider many rows rather than many more columns. 

\begin{table}[H]
\scriptsize
\centering
\caption{Mg db prediction on the targets AV, GC, VC, SE, ED. A total of 6 predictors where used in this run; radius, electronegative, van der waals volume and polarization, all for both charged and discharged materials. }
\title{Mg database on AP-RDF}
\begin{tabular}{|c|c|c|c|c|c|}
	\hline 
	$\frac{Target: \rightarrow}{Accuracy:\downarrow} $ & AV & GC & VC & SE & ED 
	 \\ 
	\hline
	$R^2$-score &  0.0284 & 0.0799 & 0.1116 &  0.0323 &  0.0540\\ 
	\hline 
	$R^2$-train & 0.40632 &   0.37274 & 0.3877 &  0.3927 & 0.3998 \\ 
	\hline
	Mean: & 2.6294&196.4378&826.694	&509.643	& 2120.2996	\\
	\hline 
	Stdev: &0.8531	&69.896	&279.4123&244.658	&990.4840	\\
	\hline 
	RMSE: &1.1041& 85.8334 &  279.413 &  244.6616 & 990.4908 \\ 
	\hline
	MAE: & 0.637 & 47.4149& 192.349 &  175.8229 & 703.5917 \\ 
	\hline
	WAPE: & 24.2602 & 24.1374 & 23.267  & 34.4992 & 33.1835 \\
	\hline
	$R^2$ CVM: &  0.0309 & 0.11039 & 0.13321  & 0.04783 & 0.05931 \\
	\hline
	components: & 6/6 & 6/6  & 5/6   & 6/6 & 6/6  \\
	\hline
\end{tabular}
\label{tab:Mg-APRDF1}
\end{table}

    
\subsubsection{Row approach to AP-RDF }%------------------------------------

The second approach gave a better correlation with the targets, here we added a new columns per value from the RDF, as explained in the method section (\ref{meth:AP-RDF2}). 

This approach seems more promising with high prediction values for several targets (GC, VC, SE, ED), and relatively low uncertainty. The correlation is stable when we switch to a db of a bigger size, which indicates that we have represented a property in a reasonable fashion. The second approach to AP-RDF will be discussed in more detail when the combined predictions are presented.  


\begin{table}[H]
\scriptsize
\centering
\caption{Mg- db prediction on the targets AV, GC, VC, SE, ED.  A total of 106 components are applicable. A total of  predictors where used in this run.}
\title{Mg database on void fraction}
\begin{tabular}{|c|c|c|c|c|c|}
	\hline 
	$\frac{Target: \rightarrow}{Accuracy:\downarrow} $ & AV & GC & VC & SE & ED 
	 \\ 
	\hline
	$R^2$-score & 0.1793 & 0.3855 & 0.3334 &  0.3711 &  0.3206\\ 
	\hline 
	$R^2$-train & 0.8741 &  0.8986 & 0.92503&  0.88771 & 0.8978 \\ 
	\hline
	Mean: 	&2.7054	&200.669&836.204	&519.0915&2250.88	\\
	\hline 
	Stdev: 	&0.4042	&25.4534	&89.270	&103.0931&395.374	\\
	\hline 
	RMSE: &0.4041& 25.464 &  89.3011 & 103.19 & 395.53 \\ 
	\hline
	MAE: & 0.3163 & 18.420& 63.552 &  77.924 & 290.96 \\ 
	\hline
	WAPE: &  11.6913 & 9.1792 & 7.6001  & 15.0115 & 12.9264 \\
	\hline
	$R^2$ CVM: & 0.2336 & 0.3956 &  0.4438  & 0.3611 & 0.32963 \\
	\hline
	components: & 18/106 & 46/106 & 35/106  & 17/106 & 35/106 \\
	\hline
\end{tabular}
\label{tab:Mg-APRDF2}
\end{table}

\begin{table}[H]
\scriptsize
\centering
\caption{Li db prediction on the targets AV, GC, VC, SE, ED. A total of 106 components are applicable. A total of  predictors where used in this run.}
\title{Li database on void fraction}
\begin{tabular}{|c|c|c|c|c|c|}
	\hline 
	$\frac{Target: \rightarrow}{Accuracy:\downarrow} $ & AV & GC & VC & SE & ED 
	 \\ 
	\hline
	$R^2$-score & 0.2894 & 0.2732 & 0.4094 &  0.2666 &  0.3280\\ 
	\hline 
	$R^2$-train & 0.8951 &  0.9032 & 0.9191 &  0.9003 & 0.9043 \\ 
	\hline
	Mean: &3.4851	&134.104	&460.5513&476.3011&1590.0499\\
	\hline 
	Stdev:&0.3428&22.71	&74.7006	&85.6635	&295.106	\\
	\hline 
	RMSE: &0.2503& 22.7127 &  74.706 & 85.6724 & 295.1143 \\ 
	\hline
	MAE: & 0.6937 & 15.74474& 52.9488 &  61.4651 & 202.1444 \\ 
	\hline
	WAPE: & 7.1808 & 11.7406 & 11.4968  & 12.90 & 12.7131 \\
	\hline
	$R^2$ CVM: & 0.3075 & 0.3379 & 0.4474  & 0.3502 & 0.3771 \\
	\hline
	components: & 42/106 & 46/106 & 44/106  & 46/106 & 46/106 \\
	\hline
\end{tabular}
\label{tab:Li-APRDF2}
\end{table}


\subsection{Combining predictors} 
When combining the predictors; msp and vnd, for the Mg db the predictions increase for AV and SE ($3\%$, $7\%$ ), while it stays relatively still for GC, VC and ED (change $ < 2\%$). The weighted absolute percentage error falls which indicates that the predictions are more reliable.



\begin{table}[H]
\scriptsize
\centering
\caption{Mg-db applying msp and vnd. A total of 66 components are applicable. Predictions on the targets; Average Voltage (AV), gravimetric capacity (GV), volumetric capacity (VC), specific energy (SE), and energy density (ED). Each row showes the number representing that type of test, as included in section (\ref{sec:evaluation_method}).}
\title{Mg database on vnd and msp}
\begin{tabular}{|c|c|c|c|c|c|}
	\hline 
	$\frac{Target: \rightarrow}{Accuracy:\downarrow} $ & AV & GC & VC & SE & ED 
	 \\ 
	\hline
	$R^2$-score 	& 0.6205 & 0.5940 & 0.66789 & 0.6710 &  0.5324\\ 
	\hline 
	$R^2$-train 	& 0.9445 & 0.9512 & 0.9611 & 0.94143 &  0.9334 \\ 
	\hline
	Mean: 	 	& 2.6786	&194.662&828.9436& 548.585& 2142.81\\
	\hline 
	Stdev:	 	& 0.2612	&18.9257	&67.7563 	&68.168	& 341.56\\
	\hline 
	RMSE: 		&0.2612& 18.9269 &  67.7746 & 68.1804 &341.58\\ 
	\hline
	MAE: 		& 0.1881 & 10.8578& 34.5623 & 49.1891 & 219.49 \\ 
	\hline
	WAPE: 		& 7.0254 & 5.57778 & 4.1694  & 8.9665 & 10.2428 \\
	\hline
	$R^2$ CVM: 	& 0.6618 	& 0.6661 	& 0.6930  	& 0.7209 &0.6424 \\
	\hline
	components: 	& 34/66 	& 35/66 	& 35/66 	 & 33/66 	&34/66 \\
	\hline
\end{tabular}
\label{tab:mg-vnd-msp}
\end{table}

When combining msp and vnd, for the Li db, the trends here are consistent with the  once from the Mg db. For AV and SE the predictions increases ($13\%$, and $11\%$). For the predictions in general, the combination of predictors either heightens the predictive score or lowers the error. 


\begin{table}[H]
\scriptsize
\centering
\caption{Li-db applying msp and vnd. A total of 123 components are applicable. Predictions on the targets; Average Voltage (AV), gravimetric capacity (GV), volumetric capacity (VC), specific energy (SE), and energy density (ED). Each row showes the number representing that type of test, as included in section (\ref{sec:evaluation_method}).}
\title{Li database on vnd and msp}
\begin{tabular}{|c|c|c|c|c|c|}
	\hline 
	$\frac{Target: \rightarrow}{Accuracy:\downarrow} $ & AV & GC & VC & SE & ED 
	 \\ 
	\hline
	$R^2$-score 	& 0.5721 & 0.6090 & 0.6824 & 0.66525 &  0.6344\\ 
	\hline 
	$R^2$-train 	& 0.9609 & 0.9533 & 0.9574 & 0.9455 &  0.9528 \\ 
	\hline
	Mean: 	 	& 3.6302	&131.3345&467.1239& 471.7594& 1664.6278\\
	\hline 
	Stdev:	 	& 0.2022	&15.6615	&54.1556 	&62.4251	& 196.7591\\
	\hline 
	RMSE: 		&0.20225& 15.6676 &  54.16726 & 62.4668 &196.8365\\ 
	\hline
	MAE: 		& 0.1474 & 10.8578& 34.45520 & 41.13432 & 135.256 \\ 
	\hline
	WAPE: 		& 4.0622 & 7.86381 & 7.37602  & 8.7193 & 8.1252 \\
	\hline
	$R^2$ CVM: 	& 0.6979 	& 0.6444 	& 0.71029 & 0.6713 &0.6590 \\
	\hline
	components: 	& 43/123 	& 44/123 	& 45/123 	 & 46/123 	&45/123 \\
	\hline
\end{tabular}
\label{tab:mg-vnd-msp}
\end{table}



\subsubsection*{Stability}

In this work we tried to predict the stability of a material, based on the same predictors as introduced, this did not work. One of several possible reasons might be that the stability of a material is connected to that particular material, and using a combination of both charged and discharged properties can confuse the ML algorithm due to the amount of unnecessary information. The predictions on charged and discharged stability can be found it the github. In addition we decided to include stability as a predictor to see if this upped our results. Using stability as a predictor did not increase the predictions over the threshold ($>2\%$) that was set.


\subsection{Combining target and predictors}
In a last effort to evaluate the method some of the targets were introduced as predictors. The idea being that if one of the targets is simple to compute through DFT, then it is possible to predict the other targets without preforming costly calculations. In essence a way to speed up high-throughput material science. First the Mg db will be considered, then the Li db and at the end a combination of different targets will briefly be looked at. 

%I.e. by introducing the voltage as a predictors for specific energy and energy density the predictions does a jump to above $80\%$. 

\begin{table}[H]%Mg msp,vnd,stab,vf
\scriptsize
\centering
\caption{Mg-db applying msp, vnd, stability and void fraction. A total of 72 components are applicable. Predictions on the targets; Average Voltage (AV), gravimetric capacity (GV), volumetric capacity (VC), specific energy (SE), and energy density (ED). Each row shows the number representing that type of test, as included in section (\ref{sec:evaluation_method}).}
\title{Mg database on vnd, msp, vf and stability}
\begin{tabular}{|c|c|c|c|c|c|}
	\hline 
	$\frac{Target: \rightarrow}{Accuracy:\downarrow} $ & AV & GC & VC & SE & ED 
	 \\ 
	\hline
	$R^2$-score 	& 0.6748 & 0.6743 & 0.6952 & 0.6786 &  0.5836\\ 
	\hline 
	$R^2$-train 	& 0.9477 & 0.9519 & 0.9618 & 0.9580 &  0.9515 \\ 
	\hline
	Mean: 	 	& 2.6624	&193.6127&826.9859& 516.2887& 2188.451\\
	\hline 
	Stdev:	 	& 0.2406	&18.3104	&68.3938 	&64.8012	& 280.0648\\
	\hline 
	RMSE: 		&0.2406& 18.3146 &  68.4006 & 64.8012 &280.0846\\ 
	\hline
	MAE: 		& 0.1719 & 10.9457& 34.9866 & 47.5026 & 192.823906 \\ 
	\hline
	WAPE: 		& 6.4566 & 5.65344 & 4.2306  & 9.2008 & 8.81097 \\
	\hline
	$R^2$ CVM: 	& 0.7072 	& 0.6805 	& 0.7206 &  0.7163 &0.62737 \\
	\hline
	components: 	& 39/72 	& 35/72 	& 37/72 	 & 34/72 	&39/172 \\
	\hline
\end{tabular}
\label{tab:mg-vnd-msp-vf-stability}
\end{table}


\begin{table}[H]%Mg msp,vnd,stab,vf
\scriptsize
\centering
\caption{Li-db applying msp, vnd, stability and void fraction. A total of 131 components are applicable. Predictions on the targets; Average Voltage (AV), gravimetric capacity (GV), volumetric capacity (VC), specific energy (SE), and energy density (ED). Each row shows the number representing that type of test, as included in section (\ref{sec:evaluation_method}).}
\title{Li database on vnd, msp, vf andstability}
\begin{tabular}{|c|c|c|c|c|c|}
	\hline 
	$\frac{Target: \rightarrow}{Accuracy:\downarrow} $ & AV & GC & VC & SE & ED 
	 \\ 
	\hline
	$R^2$-score 	& 0.6765 & 0.6367 & 0.67698 	& 0.6657 &  0.6368\\ 
	\hline 
	$R^2$-train 	& 0.9580 & 0.9491 & 0.9575 	&  0.9443 &  0.9497 \\ 
	\hline
	Mean: 	 	& 3.5426	&134.2437&460.6255& 472.3159& 1598.6273\\
	\hline 
	Stdev:	 	& 0.2164	&16.62927&54.9892	&64.09351	& 214.561\\
	\hline 
	RMSE: 		&0.2164& 16.6336 &  55.0011 	& 64.1256 &214.5891\\ 
	\hline
	MAE: 		& 0.14963 & 10.6198& 34.9866 & 42.26381 & 136.5968 \\ 
	\hline
	WAPE: 		& 4.2236 & 7.91080 & 7.70540  & 8.9482 & 8.5446 \\
	\hline
	$R^2$ CVM: 	& 0.7317 	& 0.6472 	& 0.7151	 &  0.66435 &0.6861 \\
	\hline
	components: 	& 47/131 	& 47/131 	& 49/131 	 & 51/131 	&45/131 \\
	\hline
\end{tabular}
\label{tab:Li-vnd-msp-vf-stability}
\end{table}

As a final note: If the capacity or specific energy are used as a predictor for the other targets the predictions are hovering above $97\%$ for all targets but AV. This shows that it is possible to get high enough predictions, given the right predictors, and that some vital component is lacking for the ML algorithm to achieve such an estimate. For the target average voltage, some property that is not in the capacity or specific energy, or any of the other predictors is missing. 



%\subsection{Geometrical descriptors}
%These grafs all represent the accuracy of the predictions on the training data and on new data given to the machine, with only the number density as a predictor, and the Average voltage, Gravimetric capacity, Volumetric capacity, energy density, and physical stability for the discharged material, as targets (a-f)\ref{fig:numberdensity_a-f}. Most notably the predictions on the Average voltage, Gravimetric capacity, Volumetric capacity, energy density, and specific energy are all showing a good amount of correlation, with around $60\%$ accuracy. 
%The physical stability for the discharged-,and the physical stability of the charged-materials show that there is no correlation between the number density and the physical stability. 
%It is also shown that there is no correlation between the number density and the void fraction. Or any of the other properties for that matter. 


\begin{comment}
\begin{figure}[H]
    \centering
    \begin{subfigure}{0.3\textwidth}
        \centering
        \includegraphics[width=\linewidth]{Results/2019-04-11/target=Average_Voltage.jpg}
        \caption{}
    \end{subfigure}
   % ~ 
    \begin{subfigure}{0.3\textwidth}
        \centering
        \includegraphics[width=\linewidth]{Results/2019-04-11/target=Capacity_Grav.jpg}
        \caption{}
    \end{subfigure}
       % ~ 
    \begin{subfigure}{0.3\textwidth}
        \centering
        \includegraphics[width=\linewidth]{Results/2019-04-11/target=Capacity_Vol.jpg}
        \caption{}
    \end{subfigure}
       % ~ 
    \begin{subfigure}{0.3\textwidth}
        \centering
        \includegraphics[width=\linewidth]{Results/2019-04-11/target=Specific_E_Wh_kg.jpg}
        \caption{}
    \end{subfigure}
       % ~ 
    \begin{subfigure}{0.3\textwidth}
        \centering
        \includegraphics[width=\linewidth]{Results/2019-04-11/target=E_Density_Wh_l.jpg}
            \caption{}
    \end{subfigure}
       % ~ 
    \begin{subfigure}{0.3\textwidth}
        \centering
        \includegraphics[width=\linewidth]{Results/2019-04-11/target=Stability_Discharge.jpg}
            \caption{}
    \end{subfigure}
\caption[R2 plot]{R2 plot for different runs}
    \label{fig:numberdensity_a-f}

\end{figure}

\subsection{Energy descriptions}

\begin{figure}
    \centering
    \includegraphics[width=\linewidth]{Results/plots/mean_crossvalidation_plot.png}
\caption[Cross validation and prediction uncertainty plot]{The Cross validation and the predictions uncertainty plottet for different runs with different predictors and the specific energy as the target for all runs. No removal of outliers have been implemented. l = number density, hgv = helvol, geomvol of void fraction, AV = average voltage, CGV = gravimetric and volumetric capacity, ED = energy density}
\label{fig:mean_cv}
\end{figure}
\myworries{}
\end{comment}


%Parameters.
%Volume number density
%Try to tell a STORY.


%Presentation between 30-45 minuttes. 

%Problem?
%method
%solution? 

% Test with density fraction:


% Test; weight of different parameters.















